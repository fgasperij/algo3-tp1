\section{Problema 1: Puentes sobre lava caliente}

\subsection{Presentaci\'on del problema}
%aca ponemos una interpretacion de lo que nos pide el enunciado y algunas aclaraciones de como vamos a encarar el problema.
Se quiere atravesar un puente con $n$ tablones dando saltos acotados por un valor de $x$ tablones. Se empieza afuera del puente y se pretende salir completamente de éste, es decir que como mínimo hay que saltar una vez (en el caso trivial de que $x > n$). La dificultad consiste en que ciertos tablones conocidos están rotos, y no pueden ser pisados. Lo que pide el problema es minimizar la cantidad de saltos para atravesar el puente, o aclarar que es imposible. Los puentes estarán definidos como $t_1$ $t_2$ $...$ $t_n$ donde $t_i = 0$ si el tablón está sano o $t_i = 1$ si está dañado. \\
Por ejemplo, podríamos tener el puente 0 1 0 0 con un salto máximo igual a 2. Como se arranca afuera, saltar al primer tablón se considera como un salto de 1 tablón. En este caso no podemos saltar los dos tablones permitidos porque el segundo tablón está roto (el puente, usando $X$ para marcar donde estamos parados, se vería así: X 1 0 0). El segundo salto sí podremos saltar los 2 tablones, quedando 0 1 X 0, y con el tercer salto saldremos del puente. \\
Una configuración más complicada podría ser el puente 0 0 1 0 0 0 1 1 0 0 para un salto máximo de 3 tablones, ya que ahora tenemos dos posibilidades: saltar al primer o al segundo tablón. Usaremos un algoritmo goloso para resolver el problema (saltar la mayor cantidad posible de tablones) y demostraremos que es correcto y que es la solución óptima para el problema.


\subsection{Resoluci\'on}
\subsubsection{Algoritmo}
%aca ponemos una descripcion de nuestro algorimtmo, presentamos la variables las estructuras y decimos que hacemos.
Dado este problema de optimización planteamos resolverlo con un algoritmo goloso, que consiste en seguir ''una heurística consistente para elegir la opción óptima en cada paso local con la esperanza de llegar a una solución general óptima'' [Cormen p.414 (Greedy Algorithms)].
El problema a optimizar es encontrar la mínima cantidad de saltos para cruzar el puente, y la decisi\'on golosa o la opcion \'optima en cada paso local es elegir el tablon m\'as lejano que pueda alcanzar el participante de acuerdo al rango de salto que tenga. 

El algoritmo recibe un vector con los tablones del puente (puente$[i]$) y un entero que representa el máximo salto que puede dar el participante (\textit{maxSalto}).

Teniendo esa informaci\'on inicializamos la variable \textit{actual} y \textit{proximo} en $0$, que son enteros. La primera representa en que posici\'on del puente se ubica el participante y la segunda la posici\'on del salto m\'as lejano que puede alcanzar a un tablon.
Estas variables son actualizadas por un ciclo, que en el caso que haya soluci\'on corre hasta que la posici\'on \textit{actual} sea mayor a la cantidad de tablones, es decir que el participante haya cruzado el puente.

Dentro del ciclo, se calcula la variable \textit{proximo} con una funci\'on (\textit{calcularProximoTablon}) que recibe el \textit{puente} la posici\'on \textit{actual} y el \textit{maxSalto} y prueba desde el salto m\'as largo que puede dar hasta el m\'inimo cual es el pr\'oximo tablon \'optimo, si no existe, entonces devuelve una excepci\'on y hace que el algoritmo termine o en caso contrario el ciclo lo guarda en un vector de \textit{saltos}.
\textit{Actual} se actualiza a la posici\'on \textit{proximo} en cada iteraci\'on que significa que el participante avanza en cada vuelta del ciclo.

Una vez que termina el ciclo el algoritmo devuelve el arreglo de \textit{saltos}, que es vacio si no existe soluci\'on.

\subsubsection{Pseudoc\'odigo}
%aca va el pseudocodigo del problema.
\begin{algorithm}
\begin{algorithmic}
%\STATE input: vector$<$int$>$ puente, int maxSalto 
%\STATE output: vector$<$int$>$ saltos
\STATE int cantidadTablones $\gets |puente| - 2$ \textcolor{CadetBlue}{// El vector tiene dos tablones más: tanto el primero como el último se consideran fuera del puente}
\STATE int actual $\gets 0$
\STATE int proximo $\gets 0$
\WHILE {actual $\leq$ cantidadTablones}
    \STATE proximo $\gets$ calcularProximoTablon(puente, actual, maxSalto)
    \IF {proximo $==$ $-1$}
        \RETURN vector vacío
    \ENDIF
    \STATE introducirAlFinal(saltos, proximo)
    \IF {proximo $>$ cantidadTablones}
        \RETURN saltos
    \ENDIF
    \STATE actual $\gets$ proximo
\ENDWHILE
\caption{cruzarPuente(vector$<$int$>$ puente, int maxSalto ) $\rightarrow$ vector$<$int$>$ saltos}
\end{algorithmic}
\end{algorithm}

\begin{algorithm}
\begin{algorithmic}
\STATE int cantidadTablones $\gets |puente| - 2$
\WHILE {maxSalto $>$ 0}
    \IF {actual $+$ maxSalto $>$ cantidadDeTablones}
        \RETURN cantidadDeTablones $+$ 1
    \ENDIF
    \IF {puente$[$actual $+$ maxSalto$]$ $==$ 1}
        \RETURN actual $+$ maxSalto
    \ENDIF
    \STATE maxSalto $\gets$ maxSalto $-$ 1
\ENDWHILE
\RETURN $-1$
\caption{int calcularProximoTablon(vector$<$int$>$ puente, int actual, int maxSalto )}% $\rightarrow$ int proximo}
\end{algorithmic}
\end{algorithm}
\subsection{Demostraci\'on}
%aca va la demostracion formal del problema refiriendonos al pseudocodigo o redefiniendo variables (definir todas las cosas de las que vamos a hablar).
Vamos a definir un salto como un entero natural mayor que 0 y menor que la cantidad de tablones que el participante actual puede saltar de a una sola vez.
Nuestra implementacion recorre el puente dando saltos, garantizando que en cada salto, la distancia recorrida es m\'axima. Es decir, no existe otro salto tal que la distancia desde donde estamos parados es mayor a la del salto actual y el tabl\'on en el que caes no esta roto.\\
$Distancia$ es un Nat $>$ 0. \\
$Salto$ es Nat tal que $\forall s:Salto, s > 0 \wedge s \leq Distancia$
 
Vamos a tratar de probar que dado una secuencia de saltos, si para cada salto $s$, $s$ es un "salto maximo" y si la sumatoria de saltos es mayor a la cantidad de tablones del puente, entonces nuestra secuencia es solucion del problema.

Dada un Sec$<$Salto$>$ $se$.

$(\forall i:Nat, i < se.long)(esMax(se_{i}) \wedge \sum_{j=0}^{se.long-1}se_{j} = puente.long)\implies$ \\ $\not\exists (se':Sec<Salto>) /
se.long < se'.long \wedge \sum_{j=0}^{se'.long-1}se'_{j} \geq puente.long)$  //

Llamemos a sMax a la secuencia de saltos maximos obtenida.
Supongamos que existe secuencia s de saltos tal que la cantidad de elementos de s es menor a sMax y la sumatoria de saltos es igual o mayor a la cantidad de tablones.
Bueno en particular, existe al menos un salto $s_{i}$, tal que $s_{i}$, es mayor a $sMax_{i}$,  ya que si todos los $s_{i}$, son menores a su correspondiente $sMax_{i}$,
entonces la sumatoria de sMax es mayor que la sumatoria de s. (comprobar esto ad-hoc, probablemente sale por induccion).
Bueno, supongamos que agarro el primero de todos los $s_{i}$, que es mas grande que su correspondiente $sMax_{i}$.
Hasta ese momento las dos subsecuencias (desde el principio hasta el elemento i) pesan lo mismo, entonces $s_{i}$ esta parado en el mismo lugar y hace un salto mas grande que el salto maximo ($sMax_{i}$), lo cual es absurdo.
Por lo tanto queda comprobado que ese $s_{i}$, no puede existir y la solucion es m\'axima.
\\
------------------------------------------------------------------------\\
LIMPIADA DE CARA FER\\
------------------------------------------------------------------------\\
Definimos un salto $s$ como un natural mayor a 0 y menor o igual a la distancia 
máxima que puede recorrer el participante, de sólo un salto, medida en tablones

\begin{displaymath}
	s \in Saltos \Leftrightarrow (s \in \mathbb{N}_{> 0} \land s \leq dist_{max})
\end{displaymath}

Definimos un puente como una función $p: \mathbb{N}_{>0} \to \mathbb{N}$ 
\begin{displaymath}
	p(i) = \begin{cases} 
					1 &\mbox i \leq 0 \\ 
					0 &\mbox i > \#tablones \\
					0 &\mbox i > 0 \land i \leq \#tablones \land i \in Tablones \\
					1 &\mbox i > 0 \land i \leq \#tablones \land i \notin Tablones \\
				\end{cases} %\pmod{2}
\end{displaymath}

Para cada posición $i$ del puente definimos su salto máximo $s_{max}$ como 

\begin{displaymath}
	s_{max} = max \{n \in \mathbb{N}_{>0} \mid n \leq dist_{max} \land  \neg p(n)\}
\end{displaymath}

Nuestra implementacion recorre el puente dando saltos, garantizando que en cada salto, la distancia recorrida es m\'axima. Es decir, no existe otro salto tal que la distancia desde donde estamos parados es mayor a la del salto actual y el tabl\'on en el que caes no esta roto.\\
$Distancia$ es un Nat $>$ 0. \\
$Salto$ es Nat tal que $\forall s:Salto, s > 0 \wedge s \leq Distancia$
 
Vamos a tratar de probar que dado una secuencia de saltos, si para cada salto $s$, $s$ es un "salto maximo" y si la sumatoria de saltos es mayor a la cantidad de tablones del puente, entonces nuestra secuencia es solucion del problema.

Dada un Sec$<$Salto$>$ $se$.

$(\forall i:Nat, i < se.long)(esMax(se_{i}) \wedge \sum_{j=0}^{se.long-1}se_{j} = puente.long)\implies$ \\ $\not\exists (se':Sec<Salto>) /
se.long < se'.long \wedge \sum_{j=0}^{se'.long-1}se'_{j} \geq puente.long)$  //

Llamemos a sMax a la secuencia de saltos maximos obtenida.
Supongamos que existe secuencia s de saltos tal que la cantidad de elementos de s es menor a sMax y la sumatoria de saltos es igual o mayor a la cantidad de tablones.
Bueno en particular, existe al menos un salto $s_{i}$, tal que $s_{i}$, es mayor a $sMax_{i}$,  ya que si todos los $s_{i}$, son menores a su correspondiente $sMax_{i}$,
entonces la sumatoria de sMax es mayor que la sumatoria de s. (comprobar esto ad-hoc, probablemente sale por induccion).
Bueno, supongamos que agarro el primero de todos los $s_{i}$, que es mas grande que su correspondiente $sMax_{i}$.
Hasta ese momento las dos subsecuencias (desde el principio hasta el elemento i) pesan lo mismo, entonces $s_{i}$ esta parado en el mismo lugar y hace un salto mas grande que el salto maximo ($sMax_{i}$), lo cual es absurdo.
Por lo tanto queda comprobado que ese $s_{i}$, no puede existir y la solucion es m\'axima.

\subsection{An\'alisis de complejidad}
%aca decimos cuanto cuesta cada parte del algoritmo y damos un valor final de la complejidad del algoritmo, ej O(logn).

\subsection{Test de complejidad}
%aca van los graficos y todos los testeos que hagamos para probar que en la practica el algoritmo cumple la complejidad que propusimos en el punto anterior

\subsection{Testing}
%aca ponermos todos nuestros casos bordes, como actua nuestro algoritmo en los casos particulares.
