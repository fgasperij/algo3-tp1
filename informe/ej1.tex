\section{Problema 1: Puentes sobre lava caliente}

\subsection{Presentaci\'on del problema}
%aca ponemos una interpretacion de lo que nos pide el enunciado y algunas aclaraciones de como vamos a encarar el problema.
Se quiere atravesar un puente con $n$ tablones dando saltos acotados por un valor de $x$ tablones. Se empieza afuera del puente y se pretende salir completamente de éste, es decir que como mínimo hay que saltar una vez (en el caso trivial de que $x > n$). La dificultad consiste en que ciertos tablones conocidos están rotos, y no pueden ser pisados. Lo que pide el problema es minimizar la cantidad de saltos para atravesar el puente, o aclarar que es imposible. Los puentes estarán definidos como $t_1$ $t_2$ $...$ $t_n$ donde $t_i = 0$ si el tablón está sano o $t_i = 1$ si está dañado. \\
Por ejemplo, podríamos tener el puente 0 1 0 0 con un salto máximo igual a 2. Como se arranca afuera, saltar al primer tablón se considera como un salto de 1 tablón. En este caso no podemos saltar los dos tablones permitidos porque el segundo tablón está roto (el puente, usando $X$ para marcar donde estamos parados, se vería así: X 1 0 0). El segundo salto sí podremos saltar los 2 tablones, quedando 0 1 X 0, y con el tercer salto saldremos del puente. \\
Una configuración más complicada podría ser el puente 0 0 1 0 0 0 1 1 0 0 para un salto máximo de 3 tablones, ya que ahora tenemos dos posibilidades: saltar al primer o al segundo tablón. Usaremos un algoritmo goloso para resolver el problema (saltar la mayor cantidad posible de tablones) y demostraremos que es correcto y que es la solución óptima para el problema.


\subsection{Resoluci\'on}
\subsubsection{Algoritmo}
%aca ponemos una descripcion de nuestro algorimtmo, presentamos la variables las estructuras y decimos que hacemos.
Dado este problema de optimización planteamos resolverlo con un algoritmo goloso, que consiste en seguir ''una heurística consistente para elegir la opción óptima en cada paso local con la esperanza de llegar a una solución general óptima'' [Cormen p.414 (Greedy Algorithms)].
El problema a optimizar es encontrar la mínima cantidad de saltos para cruzar el puente, y la decisi\'on golosa o la opcion \'optima en cada paso local es elegir el tablon m\'as lejano que pueda alcanzar el participante de acuerdo al rango de salto que tenga. 

El algoritmo recibe un vector con los tablones del puente (puente$[i]$) y un entero que representa el máximo salto que puede dar el participante (\textit{maxSalto}).

Teniendo esa informaci\'on inicializamos la variable \textit{actual} y \textit{proximo} en $0$, que son enteros. La primera representa en que posici\'on del puente se ubica el participante y la segunda la posici\'on del salto m\'as lejano que puede alcanzar a un tablon.
Estas variables son actualizadas por un ciclo, que en el caso que haya soluci\'on corre hasta que la posici\'on \textit{actual} sea mayor a la cantidad de tablones, es decir que el participante haya cruzado el puente.

Dentro del ciclo, se calcula la variable \textit{proximo} con una funci\'on (\textit{calcularProximoTablon}) que recibe el \textit{puente} la posici\'on \textit{actual} y el \textit{maxSalto} y prueba desde el salto m\'as largo que puede dar hasta el m\'inimo cual es el pr\'oximo tablon \'optimo, si no existe, entonces devuelve una excepci\'on y hace que el algoritmo termine o en caso contrario el ciclo lo guarda en un vector de \textit{saltos}.
\textit{Actual} se actualiza a la posici\'on \textit{proximo} en cada iteraci\'on que significa que el participante avanza en cada vuelta del ciclo.

Una vez que termina el ciclo el algoritmo devuelve el arreglo de \textit{saltos}, que es vacio si no existe soluci\'on.

\subsubsection{Pseudoc\'odigo}
%aca va el pseudocodigo del problema.
\begin{algorithm}[H]
\begin{algorithmic}[1]
%\STATE input: vector$<$int$>$ puente, int maxSalto 
%\STATE output: vector$<$int$>$ saltos
\STATE int cantidadTablones $\gets |puente| - 2$ \textcolor{CadetBlue}{// El vector tiene dos tablones más: tanto el primero como el último se consideran fuera del puente}
\STATE int actual $\gets 0$
\STATE int proximo $\gets 0$
\WHILE {actual $\leq$ cantidadTablones}
    \STATE proximo $\gets$ calcularProximoTablon(puente, actual, maxSalto)
    \IF {proximo $==$ $-1$}
        \RETURN vector vacío
    \ENDIF
    \STATE introducirAlFinal(saltos, proximo)
    \IF {proximo $>$ cantidadTablones}
        \RETURN saltos
    \ENDIF
    \STATE actual $\gets$ proximo
\ENDWHILE
\caption{cruzarPuente(vector$<$int$>$ puente, int maxSalto ) $\rightarrow$ vector$<$int$>$ saltos}
\end{algorithmic}
\end{algorithm}

\begin{algorithm}[H]
\begin{algorithmic}[1]
\STATE int cantidadTablones $\gets |puente| - 2$
\WHILE {maxSalto $>$ 0}
    \IF {actual $+$ maxSalto $>$ cantidadDeTablones}
        \RETURN cantidadDeTablones $+$ 1
    \ENDIF
    \IF {puente$[$actual $+$ maxSalto$]$ $==$ 0}
        \RETURN actual $+$ maxSalto
    \ENDIF
    \STATE maxSalto $\gets$ maxSalto $-$ 1
\ENDWHILE
\RETURN $-1$
\caption{int calcularProximoTablon(vector$<$int$>$ puente, int actual, int maxSalto )}% $\rightarrow$ int proximo}
\end{algorithmic}
\end{algorithm}
\subsection{Demostraci\'on}
%aca va la demostracion formal del problema refiriendonos al pseudocodigo o redefiniendo variables (definir todas las cosas de las que vamos a hablar).
%MAXI

Vamos a tratar de probar que dado una secuencia de saltos, si para cada salto $s$, $s$ es un "salto maximo" y si la sumatoria de saltos es mayor a la cantidad de tablones del puente, entonces nuestra secuencia es solucion del problema.

Dada un Sec$<$Salto$>$ $se$.

$(\forall i:Nat, i < se.long)(esMax(se_{i}) \wedge \sum_{j=0}^{se.long-1}se_{j} = puente.long)\implies$ \\ $\not\exists (se':Sec<Salto>) /
se.long < se'.long \wedge \sum_{j=0}^{se'.long-1}se'_{j} \geq puente.long)$  

Para probar esto, vamos a utilizar la definici\'on can\'onica de demostraci\'on de correctitud para algoritmos greedy, dada en clase por la c\'atedra. Un algoritmo goloso, puede plantearse como el siguiente esquema:
%comentar introduction to algorithim

\begin{algorithm}
\begin{algorithmic}
%\STATE input: S, F, P 
%\STATE output: S
\STATE $S^{opt}$ $=$ $\emptyset$
\WHILE {S $!=$ $\emptyset$}	
    \STATE X = $f(S)$
    \STATE S = $S - {X}$
    \IF {p($S^{opt}$,X)}
    \STATE $S^{opt} = S^{opt} \cup$ ${X}$   
    \ENDIF         
\ENDWHILE
\RETURN $S^{opt}$
\end{algorithmic}
\end{algorithm}

Luego para garantizar la correctitud del mismo hay que garantizar 4 puntos:

%comentar que esta sacada de la clase practica del dia xxxxxxx.
%esto deberia estar con indices, no con numeros cabeza

1) Definimos una noci\'on de lo que significa ser “sub-soluci\'on” de una soluci\'on.

2) Probar que $S^{opt}$ empieza siendo sub-soluci\'on de alguna soluci\'on \'optima.

3) Probar que si  $S^{opt}$ es sub-soluci\'on de alguna soluci\'on \'optima al iniciar 	una iteraci\'on del ciclo, entonces al terminar  esa iteraci\'on, $S^{opt}$ sigue siendo una subsoluci\'on de alguna soluci\'on \'optima.

4) Probar que cuando S $== \emptyset$, $S^{opt}$ es una soluci\'on \'optima.

Bueno para probar estos 4 puntos, primero vamos a definir F, S y P.


S es un multiconjunto de los saltos que uno puede dar representados por los pares (x,y) de tal que el representa un salto desde la posicion x hasta la posicion y con 0$<$ y-x$\leq$salto m\'ax.
F es una funci\'on selecciona de S el par con mayor diferencia (y - x) y menor coordenada x.
P($S^{opt}$,X) devuelve TRUE cuando el tabl\'on donde cae X no esta roto y cuando el salto no se superpone con un salto anterior.
%Puente[$\sum_{i=0}^{S^{opt}.long}S^{opt}_{j}$ + X] $==$ 0 
%Donde Puente es el array de tablones, donde un tabl\'on esta sano si su valor es 0.
Vamos a definir ser sub-soluci\'on como ser el prefijo de una soluci\'on cuando ordenamos por orden de saltos.
Como la cantidad de saltos que puede darse en un puente es finita, va existir al menos un m\'aximo local y vamos a poder asignarle un cardinal. Entonces como existe al menos un conjunto de saltos de cardinal \'optimo, el conjunto $\emptyset$ empieza siendo sub-soluci\'on de ese conjunto.
Con esto queda comprobado 1) y 2).

Dado $S^{opt}$ sub-soluci\'on al principio del ciclo, queremos garantizar que $S^{opt}$ es sub-soluci\'on al finalizarlo.
Supongamos que $S^{opt}$ es sub-soluci\'on de $S^{*}$ al principio del ciclo.
Bueno si a $S^{opt}$ no le agrego nada, sigue siendo una subsolucio\'on.
Si el caso es en el que agrego X, quiero ver que eso sigue siendo alguna sub-soluci\'on para alg\'un $S'^{*}$. En particular queremos ver que $S'^{*}$ $\subset$ $S^{*}$. 
Si agregamos X, es el caso en donde P dio true, por lo tanto X cae en un tabl\'on que no esta roto y el salto no se superpone con uno anterior.
Adem\'as X era el m\'aximo de los saltos posibles y de menor coordenada x obtenido por F.
Bueno, si X $\in$ $S^{*}$, entonces podemos tomar $S'^{*}$ = $S^{*}$, asi que veamos el caso en que X $\notin$ $S^{*}$.

Ahora tomemos $S'^{*}$ = $S^{*}$ $\cup$ ${X}$
Este conjunto puede tener elementos que se superponen, si no los tiene entonces 
$S'^{*}$ es sub-soluci\'on.
Si los tiene, y ordenamos por orden de salto (primera coordenada) , $S'^{*}$ no puede superponerse con $S^{opt}$ ya que $S^{opt}$ era sub-soluci\'on de $S^{*}$.
Por lo tanto si existe un elemento que se superponga, se superponen con X.
Sea Y un elemento que se superpone con X.
Sean $Y_{1},Y_{2},X_{1},X_{2}$ las componentes de Y,X respectivamente.
$Y_{1}\geq X_{2}$ ya que sino X no tenia la primera componente m\'inima.

Si $Y_{1} = X_{1}$ entonces $X_{2} \geq Y_{2}$ ya que sino F hubiera agarrado primero a Y. Por lo tanto yo podria tomar $S'^{*}$ = $S^{*}$ $\cup$ ${X}$ - ${Y}$ y eso seguiria siendo sub-soluci\'on.

Si $Y_{1} > X_{1}$ entonces hay dos casos:

Si $Y_{2} \leq X_{2}$ entonces todo el intervalo cubierto por Y, estaria inclu\'ido en el intervalo cubierto por X, y en el intervalo I = [$X_{1},Y_{1}$], I $\neq \emptyset$ habr\'ia que cubrirlo con un salto por lo tanto $S^{*}$ no ser\'ia subsoluci\'on.

Si $Y_{2} > X_{2}$ entonces yo se que X,Y no pueden estar en la misma soluci\'on(se superponen). Pero se que Y pertenece a alguna soluci\'on. Veamos que existe un intervalo que cubre X, pero no Y que seria el intervalo I = [$X_{1}$,$Y_{2}$].
Bueno pero como X contiene a todo este intervalo, existe una solucion que contiene al salto $S_{Y}$ = ($X_{1}$,$Y_{2}$). Asi que dado un soluci\'on que contiene a Y, al menos contiene un salto m\'as en ese intervalo.
Luego dado el intervalo I'=[$X_{2}$,$Y_{2}$], como X se superpone con Y, $X_{2}$ $>$ $Y_{1}$, por lo tanto existe un salto $S_{X}$ que con componentes ($X_{2}$,$Y_{2}$).
Luego si analizamos las dos sub-soluciones contienen al mismo intervalo en dos saltos, por lo que si existe una soluci\'on para que contiene a Y, existe una soluci\'on que contenga a X. 

Con esto queda comprobado 3).

Finalmente, si S $== \emptyset$ entonces no hay mas saltos, por lo tanto nuestra $S^{opt}$ recorre todo el puente.
Tenemos $S^{opt}$  que sabemos es una sub-soluci\'on de $S^{*}$. Y si $S^{opt}$ ya recorrio todo el puente, no hace falta agregarle ning\'un salto, por lo tanto $S^{opt} =S^{*}$.  

Con esto queda comprobado 4).
%MAXI
\\
------------------------------------------------------------------------\\
LIMPIADA DE CARA FER\\
------------------------------------------------------------------------\\
Definimos un salto $s$ como un natural mayor a 0 y menor o igual a la distancia 
máxima que puede recorrer el participante, de sólo un salto, medida en tablones

\begin{displaymath}
	s \in Saltos \Leftrightarrow (s \in \mathbb{N}_{> 0} \land s \leq dist_{max})
\end{displaymath}

Definimos un puente como una función $p: \mathbb{N}_{>0} \to \mathbb{N}$ 
\begin{displaymath}
	p(i) = \begin{cases} 
					1 &\mbox i \leq 0 \\ 
					0 &\mbox i > \#tablones \\
					0 &\mbox i > 0 \land i \leq \#tablones \land i \in Tablones \\
					1 &\mbox i > 0 \land i \leq \#tablones \land i \notin Tablones \\
				\end{cases} %\pmod{2}
\end{displaymath}

Para cada posición $i$ del puente definimos su salto máximo $s_{max}$ como 

\begin{displaymath}
	s_{max} = max \{n \in \mathbb{N}_{>0} \mid n \leq dist_{max} \land  \neg p(n)\}
\end{displaymath}

Nuestra implementacion recorre el puente dando saltos, garantizando que en cada salto, la distancia recorrida es m\'axima. Es decir, no existe otro salto tal que la distancia desde donde estamos parados es mayor a la del salto actual y el tabl\'on en el que caes no esta roto.\\
$Distancia$ es un Nat $>$ 0. \\
$Salto$ es Nat tal que $\forall s:Salto, s > 0 \wedge s \leq Distancia$
 
Vamos a tratar de probar que dado una secuencia de saltos, si para cada salto $s$, $s$ es un "salto maximo" y si la sumatoria de saltos es mayor a la cantidad de tablones del puente, entonces nuestra secuencia es solucion del problema.

Dada un Sec$<$Salto$>$ $se$.

$(\forall i:Nat, i < se.long)(esMax(se_{i}) \wedge \sum_{j=0}^{se.long-1}se_{j} = puente.long)\implies$ \\ $\not\exists (se':Sec<Salto>) /
se.long < se'.long \wedge \sum_{j=0}^{se'.long-1}se'_{j} \geq puente.long)$  //

Llamemos a sMax a la secuencia de saltos maximos obtenida.
Supongamos que existe secuencia s de saltos tal que la cantidad de elementos de s es menor a sMax y la sumatoria de saltos es igual o mayor a la cantidad de tablones.
Bueno en particular, existe al menos un salto $s_{i}$, tal que $s_{i}$, es mayor a $sMax_{i}$,  ya que si todos los $s_{i}$, son menores a su correspondiente $sMax_{i}$,
entonces la sumatoria de sMax es mayor que la sumatoria de s. (comprobar esto ad-hoc, probablemente sale por induccion).
Bueno, supongamos que agarro el primero de todos los $s_{i}$, que es mas grande que su correspondiente $sMax_{i}$.
Hasta ese momento las dos subsecuencias (desde el principio hasta el elemento i) pesan lo mismo, entonces $s_{i}$ esta parado en el mismo lugar y hace un salto mas grande que el salto maximo ($sMax_{i}$), lo cual es absurdo.
Por lo tanto queda comprobado que ese $s_{i}$, no puede existir y la solucion es m\'axima.

\subsection{An\'alisis de complejidad}
%aca decimos cuanto cuesta cada parte del algoritmo y damos un valor final de la complejidad del algoritmo, ej O(logn).
El algoritmo \textit{cruzarPuentes} se puede dividir en dos partes:
\begin{itemize}
\item Inicialización
\item Ciclo
\end{itemize}

En la Inicialización el algoritmo asigna 3 variables en $O(1)$, con lo cual su complejidad es $3*O(1) = O(1)$
El ciclo, como peor caso, itera hasta $n$ veces, donde $n$ es la \textit{cantidadDeTablones}. Adentro del ciclo se calcula la función \textit{calcularProximoTablon} para cada iteración. Esta nos dice el índice del próximo tablón óptimo para saltar, y a su vez es otro ciclo que se repite $k$ veces haciendo una cantidad acotada de operaciones $O(1)$, donde $k$ es la variable de entrada \textit{maxsalto} que es un valor acotado.
Luego el ciclo continua haciendo asignaciones y condicionales y devoluciones en $O(1)$.

Haciendo el calculo de complejidad obtenemos:

$O($\textit{cruzarPuentes}$) = 3*O(1) + n(O(k))$

Que es lo mismo que:

$O($\textit{cruzarPuentes}$) = O(n*k)$

Podemos ver que el algoritmo depende de $k$, es decir, del \textit{maxsalto} del participante, con lo cual podemos considerar que tiene una complejidad pseudopolinomial ya que depende de una variable de entrada, pero como sabemos que $k$ es acotado por una constante finalmente podemos concluir que es de orden $O(n)$.

\subsection{Test de complejidad}
%aca van los graficos y todos los testeos que hagamos para probar que en la practica el algoritmo cumple la complejidad que propusimos en el punto anterior

\subsection{Testing}
%aca ponermos todos nuestros casos bordes, como actua nuestro algoritmo en los casos particulares.
