\section{Apéndice: Códigos fuente}

\subsection{Problema 1: Puentes sobre lava caliente}

Código del algoritmo que resuelve el problema:
\lstset{language=C++,
                keywordstyle=\color{blue},
                stringstyle=\color{red},
                commentstyle=\color{magenta},
                morecomment=[l][\color{magenta}]{\#}
}
\begin{lstlisting}[frame=single]
vector<int> cruzarPuente(int puente[], int cantidadDeTablones, int largoSalto)
{
    int actual = 0, proximo = 0;
    vector<int> saltos;
    while (actual <= cantidadDeTablones) {
        proximo = calcularProximoTablon(puente, cantidadDeTablones, actual, largoSalto);

        if (proximo == IMPOSIBLE) return vector<int>();

        saltos.push_back(proximo);

        if (proximo > cantidadDeTablones) return saltos;

        actual = proximo;
    }
}

// Dado un puente y un tablón devuelve cuál es el salto más largo que se puede realizar desde ese tablón. Si no puede realizarse ningún salto devuelve IMPOSIBLE(-1).
int calcularProximoTablon(int puente[], int cantidadDeTablones, int actual, int largoSalto)
{
    while (largoSalto > 0) {
       if (actual + largoSalto > cantidadDeTablones) 
           return cantidadDeTablones+1;
       
       if (puente[actual+largoSalto] == 0)
           return actual+largoSalto;
           
       largoSalto--;
    }

    return IMPOSIBLE;
}
\end{lstlisting}

Por una cuestión de tiempos, la implementación testeada de este ejercicio está en el lenguaje Java. El algoritmo es similar a su implementación en C++.


Código de la implementaci\'on (en Java):
\lstset{language=Java,
                keywordstyle=\color{blue},
                stringstyle=\color{red},
                commentstyle=\color{magenta},
                morecomment=[l][\color{magenta}]{\#}
}
\begin{lstlisting}[frame=single]
int saltos = 0;
StringBuilder builder = new StringBuilder();
if (saltoMax > 0) {
    int pos = 0;
    while (pos < puente.largo() + 1) {
        int i = saltoMax;
        boolean salte = false;
        while (i > 0 && !salte) {
            if (puente.estaRoto(i + pos)) {
                i--;
            } else {
                pos += i;
                builder.append(" " + pos);
                saltos++;
                salte = true;
                i = saltoMax;
            }
        }
        //fallo algun intervalo
        if (!salte) {
            return "no";
        }
    }
}
if (saltos != 0) {
    //existe sol        
    return saltos + builder.toString();
}
//fallo en el primer intervalo
return "no";
\end{lstlisting}

Código de los tests de complejidad (en Java):
\lstset{language=Java,
                keywordstyle=\color{blue},
                stringstyle=\color{red},
                commentstyle=\color{magenta},
                morecomment=[l][\color{magenta}]{\#}
}
\begin{lstlisting}[frame=single]
// Este test lo vamos a utilizar para medir el tiempo que tarde en resolverse una entrada en el procesador. Para ello vamos a tratar de mediar una entrada resuelta k veces. El test va a ser similiar al de tamaño fijo pero corriendo cada instancia k veces.
public class TestDeEntradaAlProcesador {

    private static int PUENTEMAX = 500000000;

    private static Random random = new Random(new Date().getTime());

    private static int SALTOMAX = 10;

    private static int CANTIDADDECORRIDAS = 10;

    private static int CANTIDADDEIERACIONES = 10;

    public static void main(String[] args) {
        testEntradaAlProcesador();
    }

    public static void testEntradaAlProcesador() {
        for (int i = 500000; i < PUENTEMAX; i += 500000) {
            float prom = 0;
            for (int j = 0; j < CANTIDADDEIERACIONES; j++) {
                float promSubj = 0;
                int saltoMax = random.nextInt(SALTOMAX);
                boolean[] b = new boolean[i];
                for (int k = 0; k < b.length; k++) {
                    b[k] = random.nextBoolean();
                }
                for (int k = 0; k < CANTIDADDECORRIDAS; k++) {
                    Long ats = new Date().getTime();
                    Puente puente = new Puente(b);
                    Saltarin.saltar(puente, saltoMax);
                    Long dps = new Date().getTime();
                    promSubj += dps - ats;
                }
                prom += promSubj / (CANTIDADDECORRIDAS);

            }
            System.out.println(i / 100000 + "," + prom / CANTIDADDEIERACIONES);
        }
    }

}

public class TestDeTamanioFijo {

    private static int PUENTEMAX = 500000000;

    private static Random random = new Random(new Date().getTime());

    private static int SALTOMAX = 10;

    private static int CANTIDADDEIERACIONES = 100;

    // Este metodo lo utilizamos para medir tiempos, notamos que a partir de los 500000000 elementos la entrada comienza a tener un diferencia de tiempo (tF-tI) significante El primer valor es el N, el segundo una constante k la cual hace que se resuelven k puentes de tamaño N. El algoritmo de me devuelve el promedio de solucion de esos puentes. Luego hay una tercer constante m que es el tamanio del saltoMax, la implementacion genera un saltoMax aleatorio, pero que no puede ser mayor a m. Vamos a generar casos que tengan N >> saltoMax, ya que en esos casos se ve mas clara la complejidad.
    public static void main(String[] args) {
        testDeComplejidad();
    }

    public static void testDeComplejidad() {
        for (int i = 1; i < PUENTEMAX; i += 50000) {
            float prom = 0;
            for (int j = 0; j < CANTIDADDEIERACIONES; j++) {
                int saltoMax = random.nextInt(SALTOMAX);
                boolean[] b = new boolean[i];
                for (int k = 0; k < b.length; k++) {
                    b[k] = random.nextBoolean();
                }
                Long ats = new Date().getTime();
                Puente puente = new Puente(b);
                Saltarin.saltar(puente, saltoMax);
                Long dps = new Date().getTime();
                // Esto lo hacemos para evitar outlayers
                if (j > 10) {
                    prom += dps - ats;
                }
            }
            System.out.println(i / 10000 + "," + prom
                    / (CANTIDADDEIERACIONES - 10));
        }
    }
}
\end{lstlisting}

\subsection{Problema 2: Horizontes lejanos}

Código fuente del algoritmo que resuelve el problema:
\lstset{language=C++,
                keywordstyle=\color{blue},
                stringstyle=\color{red},
                commentstyle=\color{magenta},
                morecomment=[l][\color{magenta}]{\#}
}
\begin{lstlisting}[frame=single]
struct Edificio {
    int left;
    int right;
    int h;
};

enum tipos_de_eventos {EMPIEZA, TERMINA};
struct Evento {
   Evento() : edificio(0) {};
   Evento(tipos_de_eventos t, int x, int h, int edificio) : tipo(t), x(x), h(h), edificio(edificio) {};

   bool empieza() const { return tipo == EMPIEZA; };

   int edificio;
   tipos_de_eventos tipo;
   int x;
   int h;
};

struct Punto {
    Punto(int x, int y) : x(x), y(y) {};
    int x;
    int y;
};

int main(int argc, const char *argv[])
{
    int cantidadDeEdificios;
    while(true) {
        cin >> cantidadDeEdificios;
        if (cantidadDeEdificios == 0) return 0;

        vector<Edificio> edificios(cantidadDeEdificios);
        for (int i = 0; i < cantidadDeEdificios; i++) {
            edificios[i] = Edificio();
            cin >> edificios[i].left >> edificios[i].h >> edificios[i].right;

            assert(edificios[i].left >= 0);
            assert(edificios[i].h >= 0);
            assert(edificios[i].right >= 0);
            assert(edificios[i].left < edificios[i].right);
        }

        // cada edificio va a tener un evento de empezar y uno de terminar
        vector<Evento> eventos = generarEventos(edificios);
        // los eventos son ordenados por su coordenada x de forma ascendente
        // desempata por el tipo de evento, EMPIEZA gana
        sort(eventos.begin(), eventos.end(), &comparar_eventos_por_x);

        vector<Punto> contorno = calcularContorno(eventos);

        for (int i = 0; i < contorno.size(); i++) {
            cout << contorno[i].x << " " << contorno[i].y;
            if (i < (contorno.size()-1)) cout << " ";
        }
        cout << endl;
    }

    return 0;
}

vector<Punto> calcularContorno(const vector<Evento> &eventos)
{
    // alturas de edificios abiertos
    multiset<int> abiertos;
    multiset<int>::iterator itAbiertos;
    Evento eventoActual;
    int alturaContornoActual = 0;
    vector<Punto> contorno;

    for (int i = 0; i < eventos.size(); i++) {
        eventoActual = eventos[i];
        if (eventoActual.empieza()) {
            abiertos.insert(eventoActual.h);
            if (eventoActual.h > alturaContornoActual) {
                contorno.push_back(Punto(eventoActual.x, eventoActual.h));
                alturaContornoActual = eventoActual.h;
            }
        } else {
            // lo busco primero porque sólo quiero eliminar una aparición
            // y puede que haya más de un edificio abierto con la misma
            // altura
            itAbiertos = abiertos.find(eventoActual.h); 
            abiertos.erase(itAbiertos);
            if (eventoActual.h == alturaContornoActual) {
                int tempAltura = alturaContornoActual;
                alturaContornoActual = (abiertos.empty() ? 0 : *abiertos.rbegin());
                if (tempAltura != alturaContornoActual)
                    contorno.push_back(Punto(eventoActual.x, alturaContornoActual));
            }
        }
    }
    
    return contorno; 
} 

vector<Evento> generarEventos(const vector<Edificio> &edificios)
{
    int cantidadDeEdificios = edificios.size();
    vector<Evento> eventos(cantidadDeEdificios*2);
    for (int i = 0; i < cantidadDeEdificios; i++) {
        eventos[i*2] = Evento (EMPIEZA, edificios[i].left, edificios[i].h, i);
        eventos[i*2+1] = Evento (TERMINA, edificios[i].right, edificios[i].h, i);
    }
    
    return eventos;
}

bool comparar_eventos_por_x(const Evento &a, const Evento &b)
{
    if (a.x == b.x) {
        if (a.empieza() != b.empieza()) // distinto tipo => gana el de apertura
          return a.empieza();

        if (a.empieza()) { // son los dos de apertura
          return a.h > b.h;
        } else {           // son los dos de cierre
          return a.h < b.h;
        }
    }
    
    return a.x < b.x;
}
\end{lstlisting}

Código fuente del generador de instancias:
\begin{lstlisting}[frame=single]
int main(int argc, const char *argv[])
{
    srand (time(NULL));
    int i;
    for(i=1;i<=1000; i++)
    {
        for (int j = 0; j < 10; j++)
        {
            cout << i <<"\n";
            for (int k = 0; k < i; k++)
            {
                int y = rand() % (1000) + 2;
                int x = rand() % (y-1);
                int h = rand() % (1000);
                cout << x << " " << h << " " << y << "\n";
            }
            cout << "\n";
        }
    }
    cout << "0"; 
    return 0;
}
\end{lstlisting}

Código fuente del test (sólo se modifica \textbf{main}):
\begin{lstlisting}[frame=single]
int main(int argc, const char *argv[])
{
    int cantidadDeEdificios;
    int contador = 1;
    int sumaTiempos = 0;
    int tiempoParcial = 0;
    while(true) {
        cin >> cantidadDeEdificios;
        if (cantidadDeEdificios == 0) return 0;

        vector<Edificio> edificios(cantidadDeEdificios);
        for (int i = 0; i < cantidadDeEdificios; i++) {
            edificios[i] = Edificio();
            cin >> edificios[i].left >> edificios[i].h >> edificios[i].right;

            assert(edificios[i].left >= 0);
            assert(edificios[i].h >= 0);
            assert(edificios[i].right >= 0);
            assert(edificios[i].left < edificios[i].right);
        }
        for(int j = 0; j < 10; j++){
            auto start_time = chrono::high_resolution_clock::now();//TIEMPO
            // cada edificio va a tener un evento de empezar y uno de terminar
            vector<Evento> eventos = generarEventos(edificios);
            // los eventos son ordenados por su coordenada x de forma ascendente
            // desempata por el tipo de evento, EMPIEZA gana
            sort(eventos.begin(), eventos.end(), &comparar_eventos_por_x);

            vector<Punto> contorno = calcularContorno(eventos);
            auto end_time = chrono::high_resolution_clock::now();//TIEMPO
            tiempoParcial = tiempoParcial + chrono::duration_cast<chrono::nanoseconds>(end_time - start_time).count();//TIEMPO
        }
        
        tiempoParcial = tiempoParcial / 10; //divido la corrida de la misma instancia
        int sumaTiempos = sumaTiempos + tiempoParcial;
        if(contador == 10){
            sumaTiempos = sumaTiempos / 10;
            sumaTiempos = sumaTiempos / log(cantidadDeEdificios);
            cout << cantidadDeEdificios << " " << sumaTiempos / 10;
            cout << endl;
            contador=1;
            tiempoParcial = 0;
            sumaTiempos = 0;
        }
        contador++;

    }

    return 0;
}
\end{lstlisting}

\subsection{Problema 3: Biohazard}

Código fuente del algoritmo que resuelve el problema (con la poda de ``agregar camión''):
\lstset{language=C++,
                keywordstyle=\color{blue},
                stringstyle=\color{red},
                commentstyle=\color{magenta},
                morecomment=[l][\color{magenta}]{\#}
}
\begin{lstlisting}[frame=single]
struct Camion {
    Camion() : productos(set<int>()), peligrosidad(0) {};
    Camion(int producto) : peligrosidad(0) {
        productos = set<int>();
        productos.insert(producto);
    };

    set<int> productos;
    int peligrosidad;
};

int peligrosidades[1024][1024];

int main(int argc, const char *argv[])
{
    int n, umbralDePeligrosidad;
    int i = 1;
    while(true) {
        cin >> n;
        if (n == 0) return 0;
        cout << "Instancia " << i << endl;
        i++;
        cin >> umbralDePeligrosidad;

        for (int i = 0; i < n; i++) {
            for (int j = 0; j < n; j++) {
               if (i == j) peligrosidades[i][i] = 0;
               if (i < j) cin >> peligrosidades[i][j];  
               if (i > j) peligrosidades[i][j] = peligrosidades[j][i];
            }
        }

        vector<Camion> camiones;
        camiones.push_back(Camion(0));
        vector<Camion> solucion;
        for (int i = 0; i < n; i++) {
           solucion.push_back(Camion(i)); 
        }
        bt(1, camiones, solucion, umbralDePeligrosidad, n);
        
        int elementosCamiones[n];
        set<int>::iterator it;
        for(int i = 0; i < solucion.size(); i++) {
            for (it = solucion[i].productos.begin(); it != solucion[i].productos.end(); it++) {
                elementosCamiones[*it] = i;
            }
        }

        cout << solucion.size() << " ";
        for (int i = 0; i < n; i++) {
            cout << elementosCamiones[i];
            if (i < (n-1)) cout << " ";
        }
        cout << endl;
        
    }
    return 0;
}

void bt(int producto, vector<Camion> &camiones, vector<Camion> &mejorTemp,
    int umbral, int n)
{
    if (producto == n && camiones.size() < mejorTemp.size()) {
        mejorTemp = camiones;
        return;
    }
    if (producto == n) return;
    for (int j = 0; j < camiones.size(); j++) {
        if (agregarNoSuperaUmbral(producto, camiones[j], umbral)) {
            int peligrosidadPrevia = agregarProducto(producto, camiones[j]);
            bt(producto+1, camiones, mejorTemp, umbral, n);
            sacarProducto(producto, camiones[j], peligrosidadPrevia);
        }
    }
    if (camiones.size() < (mejorTemp.size()-1)) {
        camiones.push_back(Camion(producto));
        bt(producto+1, camiones, mejorTemp, umbral, n);
        camiones.pop_back();
    }

    return;
}

bool agregarNoSuperaUmbral(int producto, const Camion &camion, int umbral)
{
    int acumPeligrosidad = camion.peligrosidad;
    set<int>::iterator it;
    for (it = camion.productos.begin(); it != camion.productos.end(); it++) {
        acumPeligrosidad += peligrosidades[producto][*it];
        if (acumPeligrosidad > umbral) return false;
    }
    
    return true;
}

void sacarProducto(int producto, Camion &camion, int peligrosidadPrevia)
{
    camion.productos.erase(producto);
    camion.peligrosidad = peligrosidadPrevia;
}

int agregarProducto(int producto, Camion &camion)
{
    int peligrosidadPrevia = camion.peligrosidad;
    set<int>::iterator it;
    for (it = camion.productos.begin(); it != camion.productos.end(); it++) {
        camion.peligrosidad += peligrosidades[producto][*it];
    }
    camion.productos.insert(producto);
    
    return peligrosidadPrevia;
}
\end{lstlisting}

Código fuente del generador de instancias aleatorias:
\begin{lstlisting}[frame=single]
// Este programa va a crear una cantidad X de instancias aleatorias de N productos, para N = {1,...,algunValor} y las va a devolver por SALIDA ESTANDAR (consola). El umbral va a ser fijo. Vale que para toda peligrosidad, 0 <= peligrosidad <= UMBRAL + 1.
// Hay que usar ./randomGen > textFile para guardar a disco.
// g++ -O3 randomGen.cpp -o randomGen

#include <iostream>
#include <unistd.h>
#include <cstdlib>
#include <cstdlib>

using namespace std;

const int CANT_INSTANCIAS = 5;
const int LIM_PRODUCTOS = 15;
const int UMBRAL = 100;

int main(int argc, const char* argv[]) {
    srand(time(NULL) + getpid()); // Seedeo
    for (int totalProductos = 2; totalProductos <= LIM_PRODUCTOS; totalProductos++) {
        for (int instancia = 1; instancia <= CANT_INSTANCIAS; instancia++) {
            cout << totalProductos << " " << UMBRAL << endl;
            for (int producto = 1; producto <= totalProductos - 1; producto++) {
                for (int peligrosidad = 1; peligrosidad <= totalProductos - producto; peligrosidad++) {
                    cout << rand() % (UMBRAL + 2) << " "; // Esto genera un entero random entre 0 y UMBRAL + 1.
                }
                cout << endl;
            }
        }
    }
    cout << "0" << endl;
}
\end{lstlisting}

Código fuente del test de complejidad (sólo se modifican la función \textbf{bt} y \textbf{main}):
\begin{lstlisting}[frame=single]
// Dado un archivo de entrada con X instancias para cada N cantidad de productos (N <= 100), devuelvo por SALIDA ESTANDAR, en cada linea, lo siguiente:
// N <espacio> promedio{ los minimo{ tiempos en ejecutar la instancia i-ésima para N productos } de todas las instancias de N productos}
// Es decir, cada instancia se ejecutará una cierta cantidad de veces y se tomará el mínimo para mejorar la precisión, y luego se promediaran estos valores para las instancias de cada cantidad de productos.
// g++ -std=c++0x p3-test.cpp -o p3-test
#include <climits>
#include <iostream>
#include <vector>
#include <set>
#include <chrono>

using namespace std;

const int MAX_CANTIDAD_PRODUCTOS = 100;
const int FOR_MINIMO = 2;
const bool PODA_AGREGAR_CAMION = true;
const bool PODA_RESTO_PRODUCTOS = true;

int peligrosidades[MAX_CANTIDAD_PRODUCTOS][MAX_CANTIDAD_PRODUCTOS];

int main(int argc, const char *argv[])
{   
    int sumaTiemposPorCantProductos[MAX_CANTIDAD_PRODUCTOS + 1] = { }; // Pongo cada tiempo en su casilla correspondiente, ie tiempo para i productos en suma[i]
    int cantidadInstanciasPorCantProductos[MAX_CANTIDAD_PRODUCTOS + 1] = { }; // Idem para la cantidad de instancias, esto lo uso para promediar
    int n, umbralDePeligrosidad;
    while(true) {
        cin >> n;
        if (n == 0) break;
        cin >> umbralDePeligrosidad;

        for (int i = 0; i < n; i++) {
            for (int j = 0; j < n; j++) {
               if (i == j) peligrosidades[i][i] = 0;
               if (i < j) cin >> peligrosidades[i][j];  
               if (i > j) peligrosidades[i][j] = peligrosidades[j][i];
            }
        }
        
        chrono::microseconds minTiempo(INT_MAX);
        for (int i = 0; i < FOR_MINIMO; i++) {
            // Acá considero que arranca el algoritmo
            auto start_time = chrono::high_resolution_clock::now();
            vector<Camion> camiones;
            camiones.push_back(Camion(0));
            vector<Camion> solucion;
            for (int i = 0; i < n; i++) {
               solucion.push_back(Camion(i)); 
            }
            bt(1, camiones, solucion, umbralDePeligrosidad, n);
            // Acá terminó
            auto end_time = chrono::high_resolution_clock::now();
            chrono::microseconds temp = chrono::duration_cast<chrono::microseconds>(end_time - start_time);
            if (temp < minTiempo)
                minTiempo = temp;
        }
        sumaTiemposPorCantProductos[n] += minTiempo.count();
        cantidadInstanciasPorCantProductos[n]++;
    }
    
    for (int i = 2; i <= MAX_CANTIDAD_PRODUCTOS; i++) {
        int instancias = cantidadInstanciasPorCantProductos[i];
        if (instancias == 0) {
            continue;
        } else {
            cout << i << " " << sumaTiemposPorCantProductos[i] / instancias << endl;
        }
    }
    return 0;
}

void bt(int producto, vector<Camion> &camiones, vector<Camion> &mejorTemp,
    int umbral, int n)
{
    if (producto == n && camiones.size() < mejorTemp.size()) {
        mejorTemp = camiones;
        return;
    }
    if (producto == n) return;
    for (int j = 0; j < camiones.size(); j++) {
        if (agregarNoSuperaUmbral(producto, camiones[j], umbral)) {
            int peligrosidadPrevia = agregarProducto(producto, camiones[j]);
            if (PODA_RESTO_PRODUCTOS) { // PODA DE RESTO DE LOS PRODUCTOS //
                if (camiones.size() == mejorTemp.size() - 1) {
                    //cout << "PODA PRODUCTOS" << endl;
                    for (int i = producto + 1; i < n; i++) {
                        bool entraEnAlguno = false;
                        for (int c = 0; c < camiones.size(); c++) {
                            if (agregarNoSuperaUmbral(i, camiones[c], umbral)) {
                                entraEnAlguno = true;
                                break;
                            }
                        }
                        if (!entraEnAlguno) {
                            sacarProducto(producto, camiones[j], peligrosidadPrevia);
                            continue;
                        }
                    }
                }
            }
            bt(producto + 1, camiones, mejorTemp, umbral, n);
            sacarProducto(producto, camiones[j], peligrosidadPrevia);
        }
    }
    if (PODA_AGREGAR_CAMION) { // PODA DE AGREGAR CAMION //
        if (camiones.size() == mejorTemp.size() - 1) {
            //cout << "PODA CAMIONES" << endl;
            return;
        }
    }
    camiones.push_back(Camion(producto));
    bt(producto+1, camiones, mejorTemp, umbral, n);
    camiones.pop_back();

    return;
}
\end{lstlisting}
