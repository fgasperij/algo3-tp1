\paragraph{Edificio}
Definiremos un edificio $e$ como una tupla $<e_{principio}$, $e_{final}$, $e_{altura}>$, la escribiremos de la siguiente forma por ser más compacta $<e_p$, $e_f$, $e_h>$, que cumple la siguiente propiedad:
\begin{displaymath}
	e_p, e_f, e_h \in \mathbb{N} \land e_p > 0 \land e_p < e_f \land e_h > 0
\end{displaymath}
La misma representa a un edificio $e$ que empieza en $e_p$, tiene una altura de $e_h$ y termina en $e_f$. 
Representados bidimensionalmente sobre un eje cartesiano los valores $e_p$ y $e_f$ se corresponden con
sus coordenadas sobre el eje de las abscisas $X$.

\paragraph{Evento}
Definiremos un evento como una tupla $<ev_x$, $ev_h$, $ev_t>$ que cumple la siguiente propiedad:
\begin{displaymath}
	ev_x, ev_h, ev_t \in \mathbb{N} \land ev_t \in \{ 1, 0\}
\end{displaymath}
$ev_x$ representa el valor de la coordenada $x$ del evento, $ev_h$ la altura del evento y $ev_t$ el tipo de 
evento, 1 para un evento que representa el comienzo de un edificio y 0 para los eventos que representan el fin de un edificio.
Dado un edificio $e = <e_p$, $e_f$, $e_h>$  diremos que dos eventos lo representan $ev_{abrir} = <e_p$, $e_h$, $1>$ y $ev_{cerrar} = <e_f$, $e_h$, $0>$.
Es decir, que cada edificio lo representamos con dos eventos, uno que marca el comienzo y otro que marca el fin del mismo. 
Dado un conjunto $E$ de edificios decimos que el conjunto $Ev$ de eventos que lo representa es aquel que contiene dos eventos
por cada edificio de $E$ derivados de la forma antes mostrada.

\paragraph{Contorno}
Dado un conjunto de edificios $E$ definiremos una función $h$ que toma un $x \in \mathbb{N}$ y un conjunto de edificios:
\begin{displaymath}
	h: (x,E) \to \mathbb{N} 
\end{displaymath}
$h$ nos devuelve la altura del edificio más alto que cubre ese valor de $x$ 
\begin{displaymath}
	h(x, E) = \begin{cases} 
						0 & \nexists e \in E | (x \geq e_p \land x < e_f) \\
						max\{ e_h | e \in E  \land e_p \leq x \land x < e_f \} & \exists e \in E | (x \geq e_p \land x < e_f)
				\end{cases} %\pmod{2}
\end{displaymath}
contando con $h$ ya podemos definir correctamente nuestro conjunto de puntos que conforman el contorno del
conjunto de edificios $E$:
\begin{displaymath}
	C = \{ p \in \mathbb{N}^2 | h(p_x - 1, E) \neq h(p_x, E) \land (\exists ev \in Ev | ev_x = p_x \land ev_h = p_y) \}
\end{displaymath}
Es decir, los puntos que pertenezcan al conjunto $contorno$ registrarán los cambios de altura. Cada punto
del contorno puede interpretarse como: a partir de este $p_x^i$ y hasta el $p_x^{i+1}$ el edificio con mayor
altura en todo el rango tiene altura igual a $p_y^i$ (asumiendo que los $p^i$ del contorno están ordenados
de forma creciente por su coordenada $x$). Los cambios de altura pueden producirse
porque un edificio empiece o porque un edificio haya concluido. 
Dado un conjunto de edificios $E$ existe un único conjunto $C$ que representa su contorno. 
Vale la pena aclarar que por como está planteado el problema la función $h$ nos dice que si tenemos un conjunto $E$
de edificios que sólo contiene un edificio $e = <e_p, e_f, e_h>$ entonces $h(e_p, E) = e_h$ pero $h(e_f, E) = 0$.
La altura de los edificio representa de esta forma un intervalo cerrado a izquierda pero abierto a derecha
respecto a la información que el contorno registra.

\textbf{Correctitud}
\par
A continuación se presenta el pseudocódigo de nuestra implementación sobre el cual plantearemos la
demostración de correctitud:

De ahora en más vamos a referirnos como $contorno$ a la secuencia que contiene a todos los elementos de
el conjunto de contorno $C$ ordenados por su coordenada $p_x^i$. Además, vamos a referirnos a que un
edificio $e$ contiene o abarca a un valor $x$ cuando $x \in [e_p, e_f)$.

Vamos a demostrar algunos lemas que nos servirán para realizar la demostración.

\paragraph{Lema 1}
Todos los puntos pertenecientes a $C$ satisfacen:
\begin{enumerate}
	\item $(\forall p \in C) \exists ev \in Eventos \text{ / } p_x = ev_x$
\end{enumerate}
La propiedad se desprende de la misma definición del conjunto $contorno$. Como $h(p_x, Edificios) \neq h(p_x - 1, Edificios)$ 
necesariamente ese cambio en la función de altura tiene que haberse producido porque un edificio 
$e = <e_p$, $p_x$, $h(p_x-1, Edificios)>$ concluyó, en el caso de que $h(p_x, Edificios) < h(p_x - 1, Edificios)$, o porque un
edificio $e = <p_x$, $e_f$, $h(p_x, Edificios)>$ comenzó, en el caso de que $h(p_x, Edificios) > h(p_x - 1, Edificios)$, ya que 
la altura por defecto para cualquier valor de $x$ que no esté contenido por un edificio del conjunto de edificios es 0. En los dos
casos tenemos un edificio que empieza o termina en $p_x$. Como el conjunto $Eventos$ contiene precisamente un evento por cada comienzo
y fin de un edificio, en particular contiene a los que satisfacen en cada caso la propiedad expresada.

\paragraph{Lema 2}
Sea S la secuencia de eventos tal que:
\begin{displaymath}
	ev \in S \Leftrightarrow ev \in Eventos
\end{displaymath}
de forma tal que el orden en el que se encuentran es el impuesto por agregarlos en el orden en el que la función 
proximo\_evento($Eventos$) los devuelve. Se cumple que:
\begin{displaymath}
	(\forall ev^1, ev^2 \in Eventos \text{ / } ev^1_x = ev^2_x) ev^1_h * ev^1_t > ev^2_h * ev^2_t \Rightarrow posicion(ev^1, S) < posicion(ev^2, S)
\end{displaymath}
Los casos que podemos tener son los siguientes:
\begin{enumerate}
	\item $\text{abre}(ev^1) \land \neg\text{abre}(ev^2)$  en este caso vemos que necesariamente $ev^1$ va a ser devuelto
	primero por proximo\_evento() porque siempre si los eventos comparten la coordenada $x$ prioriza los de apertura
	con la función filtrarLosDeCierre(Eventos).
	\item $\text{abre}(ev^1) \land \text{abre}(ev^2)$  podemos ver en el algoritmo de proximo\_evento() que cuando luego
	de filtrar por $min_x$ y quedarse con los de apertura en caso de que existan, estamos en el caso de que existen,
	$ev_1$ y $ev_2$ cumplen, devuelve primero el de mayor altura. Con lo cual la posición del de mayor altura
	necesariamente es menor.
\end{enumerate}
$\square$

\paragraph{Lema 3}
Sea $contorno$ la secuencia de puntos del contorno $C$ ordenados por su coordenada $x$. La secuencia tiene 
$n+1$ puntos $p^0,...,p^n$:
\begin{displaymath}
	(\forall x \in \mathbb{N} \text{ / } x < p_x^0 \lor x \geq p_x^n) h(x, E) = 0
\end{displaymath}
y además:
\begin{displaymath}
	(\forall x \in \mathbb{N} \text{ / } x \geq p_x^0 \land x < p_x^n) h(x, E) = p_y^{maxXmenor}
\end{displaymath}
con $p_y^{maxXmenor}$ igual al $p_y$ del $p$ que tiene el máximo $p_x$  de todos los $p$ que cumplen $p_x \leq x$.

La primer propiedad predica que antes del primer edificio y después del último, asumiendo un orden por coordenada $x$,
la función $h$ siempre vale 0. Estose desprende directamente de su definición, si todavía no empieza ningún edificio o
si ya terminaron todos es imposible que la altura valga algo mayor a 0 porque por defecto vale 0.
La segunda propiedad nos dice que entre dos puntos $p^1$, $p^2$ consecutivos del contorno la función $h$ necesariamente tiene
que devolvernos $p_y^1$. Si no fuera así entonces existiría un $x$ entre $p_x^1$ y $p_x^2$ en el cual $h(x, Edificios)$ nos devuelve
distinto de $p_y^1$. Como $p_y^1 = h(p^1_x, Edificios)$, por definición de contorno debería existir un punto $p$ del mismo
tal que $p_x = x$. Pero $x$ se encuentra entre $p_x^1$ y $p_x^2$ que son puntos consecutivos del contorno por lo cual no
puede existir ningún cambio de altura entre ellos ya que si existiera debería haber un punto del contorno allí y eso es
absurdo porque presumimos que eran consecutivos.
$\square$
 
Nuestro algoritmo consiste en un ciclo que itera sobre todos los elementos de $Eventos$. Vamos a demostrar
que el ciclo cumple el siguiente invariante:
\begin{displaymath}
	esPrefijo(contorno_{parcial}, contorno) 
\end{displaymath}

\textbf{Antes de entrar al ciclo}
\par
Antes de entrar al ciclo $contorno_{parcial} = secuencia\_vacia$. La secuencia vacía es prefijo de todas las secuencias, por lo tanto
el invariante se cumple antes de entrar al ciclo por primera vez.
\par
\textbf{El ciclo preserva el invariante}
\par
Si el ciclo no agrega puntos a $contorno_parcial$ entonces necesariamente sigue siendo prefijo de $contorno$ por lo cual
el invariante se preserva. Necesitamos demostrar que lo preserva cuando agrega elementos al mismo. Llamamos $p^i$ al último 
elemento de la secuencia $contorno_{parcial}$ y $h_{actual}$ a $p^i_y$. Existen dos casos en los que agrega elementos a 
$contorno_parcial$, estos son cuando se cumplen alguna de estas dos proposiciones:
\begin{description}
	\item[Caso 1] $\text{abre}(ev) \land h_{actual} < ev_h$
	\item[Caso 2] $\neg\text{abre}(ev) \land ev_h = h_{actual} \land maxAbierto(ev_x) < h_{actual}$
\end{description}
$contorno_{parcial}$ es prefijo de $contorno$. Para que $contorno_{parcial}$ continue siendo prefijo el elemento
que agregue tiene que ser necesariamente $p^{i+1}$ en la secuencia $contorno$. $p^{i+1}_x$ es el mínimo $x$ tal que $x > p^i_x$
 y $h(x, Edificios) \neq p^i_y$. Por \textbf{Lema 1} sabemos que necesariamente existe un $ev$ tal que $ev_x = p^{i+1}_x$.
\par
Si $p^{i+1}_x > p^{i}_x$ entonces vale abre($ev$). Por \textbf{Lema 2} sabemos que nuestro algoritmo
iterará primero sobre el $ev^i$ con $ev^i_x = p^{i+1}_x \land \text{abre}(ev^i)$ de mayor altura. Entonces, necesariamente itera
primero sobre el que cumple que $ev_x = p^{i+1}_x \land h_{actual} < ev_h$. Como entra en el Caso 1 sabemos que lo va a agregar y además
sabemos que como proximo\_evento() devuelve primero los que menor coordenada $x$ tengan será el con mínimo $ev_x$ que cumple el Caso 1.
Entonces concluimos que si $p^{i+1}_x > p^{i}_x$ entonces nuestro algoritmo lo va a agregar a contorno parcial y va a ser el
primero que agregue. Por lo tanto, $contorno_{parcial}$ sigue siendo prefijo de $contorno$.
\par
El otro caso es cuando vale $p^{i+1}_x < p^{i}_x$. Ésto implica que $\neg\text{abre}(ev)$, tomando $ev$ con el que se 
corresponde con $p^{i+1}$. $ev_h = h_{actual}$ se desprende de que si fuera mayor caemos en un absurdo porque asumimos que 
$p^{i+1}_x < p^{i}_x$ y si fuera menor entonces no cambiaría la altura, lo cual lo excluiría de la definición de los 
puntos pertenecientes a $C$. Por último, es necesario que $maxAbierto(ev_x) < h_{actual}$ porque si no se cumpliera
esto no existiría tal cambio de altura, por lo cual el punto no pertenecería la contorno. Por lo tanto, el próximo punto
del contorno cae en nuestro Caso 2 y también es agregado al $contorno_{parcial}$ para que preserve la propiedad
de ser prefijo de $contorno$.
\par
Finalmente, necesitamos demostrar que la satisfacción del invariante y la negación de la guarda del ciclo implican que 
$contorno_{parcial} = contorno$. Veamos que:
\begin{enumerate}
	\item Comparten el último elemento: el último elemento de $contorno$ cumple que $p_y = 0 \land p_x = e_f^max$ con $e_f^max$
	el maximo valor de finalizacion de entre todos los edificios. Antes de $e_f^max$ $h \geq e_h^max$ y para todo $x$ mayor o igual
	a $e_x^max$ valdrá 0. Por lo tanto, es el último punto donde cambia la altura, será el último punto de mi contorno. Ese punto
	además tiene un evento correspondiente que es el último que recorre proximo\_evento() y cumple el Caso 2 por lo
	cual también será agregado.
	\item El orden que guardan sus elementos es total: esto ya lo demostramos, es simplemente que $p^i_x < p^{+1}i_x$ para 
	todos los i entre 0 y las longitud de $contorno$. 
\end{enumerate}
Estas dos propiedades y el hecho de que sea prefijo una de otra nos implica directamente que las secuencias son iguales. Por
lo tanto, al concluir el ciclo, efectivamente $contorno_{parcial} = contorno$.