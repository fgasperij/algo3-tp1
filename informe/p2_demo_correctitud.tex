\paragraph{Edificio}
Definiremos un edificio $e$ como una tupla $<e_{principio}$, $e_{final}$, $e_{altura}>$, la escribiremos de la siguiente forma por ser más compacta $<e_p$, $e_f$, $e_h>$, que cumple la siguiente propiedad:
\begin{displaymath}
	e_p, e_f, e_h \in \mathbb{N} \land e_p > 0 \land e_p < e_f \land e_h > 0
\end{displaymath}
La misma representa a un edificio $e$ que empieza en $e_p$, tiene una altura de $e_h$ y termina en $e_f$. 
Representados bidimensionalmente sobre un eje cartesiano los valores $e_p$ y $e_f$ se corresponden con
sus coordenadas sobre el eje de las abscisas $X$.

\paragraph{Evento}
Definiremos un evento como una tupla $<ev_x$, $ev_h$, $ev_t>$ que cumple la siguiente propiedad:
\begin{displaymath}
	ev_x, ev_h, ev_t \in \mathbb{N} \land ev_t \in \{ abrir, cerrar\}
\end{displaymath}
$ev_x$ representa el valor de la coordenada $x$ del evento, $ev_h$ la altura del evento y $ev_t$ el tipo de 
evento.
Dado un edificio $e = <e_p$, $e_f$, $e_h>$  diremos que dos eventos lo representan $ev_{abrir} = <e_p$, $e_h$, $abrir>$ y $ev_{cerrar} = <e_f$, $e_h$, $cerrar>$.
Es decir, que cada edificio lo representamos con dos eventos, uno que marca el comienzo y otro que marca el fin del mismo. 
Dado un conjunto $E$ de edificios decimos que el conjunto $Ev$ de eventos que lo representa es aquel que contiene dos eventos
por cada edificio de $E$ derivados de la forma antes mostrada.

\paragraph{Contorno}
Dado un conjunto de edificios $E$ definiremos una función $h$ que toma un $x \in \mathbb{N}$ y un conjunto de edificios:
\begin{displaymath}
	h: (x,E) \to \mathbb{N} 
\end{displaymath}
$h$ nos devuelve la altura del edificio más alto que cubre ese valor de $x$ 
\begin{displaymath}
	h(x, E) = \begin{cases} 
						0 & \nexists e \in E | (x \geq e_p \land x < e_f) \\
						max\{ e_h | e \in E  \land e_p \leq x \land x < e_f \} & \exists e \in E | (x \geq e_p \land x < e_f)
				\end{cases} %\pmod{2}
\end{displaymath}
contando con $h$ ya podemos definir correctamente nuestro conjunto de puntos que conforman el contorno del
conjunto de edificios $E$:
\begin{displaymath}
	C = \{ p = (x_p, y_p) | h(x_p - 1, E) \neq h(x_p, E) \land (\exists ev \in Ev | ev_x = x_p \land ev_h = y_p) \}
\end{displaymath}
Es decir, los puntos que pertenezcan al conjunto $contorno$ registrarán los cambios de altura. Cada punto
del contorno puede interpretarse como: a partir de este $p_x^i$ y hasta el $p_x^{i+1}$ el edificio con mayor
altura en todo el rango tiene altura igual a $p_y^i$ (asumiendo que los $p^i$ del contorno están ordenados
de forma creciente por su coordenada $x$). Los cambios de altura pueden producirse
porque un edificio empiece o porque un edificio haya concluido. 
Dado un conjunto de edificios $E$ existe un único conjunto $C$ que representa su contorno. 
Vale la pena aclarar que por como está planteado el problema la función $h$ nos dice que si tenemos un conjunto $E$
de edificios que sólo contiene un edificio $e = <e_p, e_f, e_h>$ entonces $h(e_p, E) = e_h$ pero $h(e_f, E) = 0$.

\paragraph{Algoritmo} 
A continuación se presenta el pseudocódigo de nuestra implementación sobre el cual plantearemos la
demostración de correctitud:

\begin{algorithm}[H]
\begin{algorithmic}
\STATE $eventos$ $\gets$ extraer\_eventos($edificios$)
\STATE $abiertos$ $\gets$ $\emptyset$			// permite elementos repetidos
\STATE $h_{actual}$ $\gets$ 0
\WHILE {$eventos$ $\neq$ $\emptyset$}
	\STATE $ev$ $\gets$ proximo\_evento($eventos$)
	\IF {abre($ev$)}
		\STATE Agregar($abiertos$, $ev_h$)
		\IF {$h_{actual}$ $<$ $ev_h$}
			\STATE $h_{actual}$ $\gets$ $ev_h$
			\STATE Agregar($<ev_x, ev_h>$, $contornoParcial$)
		\ENDIF
	\ELSE
		\STATE Sacar($abiertos$, $ev_h$)			// quita una sola repetición del elemento
		\IF {$h_{actual}$ = $ev_h$ $\land$ max($abiertos$) $<$ $h_{actual}$}
			\STATE $h_{actual}$ = max($abiertos$)
			\STATE Agregar($<ev_x, h_{actual}>$, $contornoParcial$)  
		\ENDIF
	\ENDIF
\ENDWHILE
\caption{horizontes\_lejanos}
\end{algorithmic}
\end{algorithm}

\begin{algorithm}[H]
\begin{algorithmic}
	\STATE $min_x$ = min( $\{ ev_x | ev \in eventos\}$ )
	\STATE $comparten\_min_x$ = $\{ ev \in eventos | ev_x = min_x \}$
	\IF {quedanEventosDeApertura($comparten\_min_x$)}
		\STATE $comparten\_min_x$ = filtrarLosDeCierre($comparten\_min_x$)
		\RETURN mayorAltura($comparten\_min_x$)
	\ELSE
		\RETURN menorAltura($comparten\_min_x$)
	\ENDIF
\caption{proximo\_evento}
\end{algorithmic}
\end{algorithm}

(TODO hacer que el código desempate de la misma forma)

El invariante que cumple el ciclo de nuestro algoritmo, tomando $ev = proximo\_evento(eventos)$, comprende las siguientes
tres propiedades simultaneamente:
\begin{enumerate}
	\item $h_{actual} = h(ev_x^{i-1}, E)$
	\item $abiertos = \{ e_h | e \in Edificios \land e_p \leq (ev_x^i) \land e_f > (ev_x^i) ) \}$
	\item $contornoParcial = \{ p \in \mathbb{N}^2 | p \in C \land p_x < ev_x \}$
\end{enumerate}
(TODO LEMA: mostrar que la altura no puede cambiar entre dos eventos consecutivos de proximo\_evento)

Primero mostraremos que el invariante se cumple antes de entrar al ciclo, con $ev = proximo\_evento(Eventos)$:
\begin{description}
	\item[$h_{actual}$] antes de entrar al ciclo el conjunto $Eventos$ posee todos los eventos derivados
	del conjunto de edificios. Por lo tanto, $ev$ se va a corresponder con alguno de los eventos que 
	tengan $ev_x = min_x$ y $ev_x - 1$ va a ser un valor que no puede estar cubierto por ningún edificio
	ya que es previo al comienzo de cualquiera. Como $h$ es igual a 0 para todo $x$ que no esté contenido
	dentro del intervalo cerrado-abierto definido por un edificio se cumple que $h_{actual} = h(ev_x - 1, E)$.
	
	\item[$abiertos$] antes de entrar al ciclo el multiconjunto $abiertos = \emptyset$. Esto es correcto porque antes de
	entrar al ciclo $ev_x - 1$ se corresponde con una ubicación previa al comienzo de algún edificio. Por lo tanto,
	no puede haber ningún edificio abierto.
	
	\item[$contornoParcial$] el conjunto $contornoParcial$ comienza vacío. Como $ev_x$ es un valor de $x$ previo al comienzo del edificio
	sería absurdo que hubiera un punto que perteneciera al conjunto $contorno$ del conjunto $Edificios$ y tuviera
	un $x$ menor estricto al $e_p$ de todos los edificios. La única forma de que ocurra un cambio de altura, que es
	como están definidos los puntos del contorno, es que empiece o termine un edificio. Como antes de $ev_x$ todavía 
	no empezó ni terminó ningún edificio no puede haber ningún punto que pertenezca al conjunto $contorno$ y 
	tenga un $p_x < ev_x$.
\end{description}
luego debemos mostrar que si el invariante se cumple antes de entrar al ciclo entonces necesariamente seguirá 
cumpliéndose luego de ejecutarse el mismo. Vamos a suponer que estamos en la iteracion $i$. 
Primero veamos que sucede con el multiconjunto $abiertos$. Notemos que
$abiertos$ es modificado en todas las iteraciones del ciclo. Lo único que cambia es que cuando $ev$ es de tipo
Abrir, vale $ev_t$, entonces se le agrega un elemento y cuando es de tipo Cerrar, vale $\neg ev_t$, se le elimina un elemento. 
Entonces tenemos dos casos:
\begin{enumerate}
	\item vale $ev_t$, el evento es de tipo Abrir: en este caso $ev_h$ es agregado al multiconjunto $abiertos$. 
	Si $abiertos$ contenia las alturas de todos los edificios abiertos en $ev^{i-1}_x$ y ahora contiene $ev^{i}_h$
	porque vale $ev_t$, la agregamos en esta iteracion $i$, entonces necesariamente contiene todas las alturas
	de los edificios abiertos hasta $ev^{i+1}_x$ porque entre dos eventos consecutivos (por proximo\_evento()) no 
	pueden ni empezar ni terminar edificios. (TODO quizas haga falta un lema mostrando esto de que entre dos eventos
	consecutivos no puede "suceder" nada)
	
	\item no vale $ev_t$, el evento es de tipo Cerrar: en este caso $ev_h$ es eliminado una vez del multiconjunto
	$abiertos$. $abiertos$ contenia todas las alturas de todos los edificios abiertos en $ev^{i-1}_x$ y en esta
	iteracion le eliminamos la del unico edificio que termina en el intervalo $[ev^{i-1}_x,ev^{i+1}_x]$. Como
	no puede haber cerrado otro ni se pudo haber abierto otro en ese intervalo entonces contiene todas las 
	alturas de todos los edificios abiertos en $(ev^{i+1}_x-1)$ 
\end{enumerate}











\begin{description}
	\item[$h_{actual}$] 
	
	\item[$abiertos$] 
	
	\item[$contornoParcial$] 
\end{description}





