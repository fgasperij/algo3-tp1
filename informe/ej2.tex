\section{Problema 2: Horizontes lejanos}

\subsection{Presentaci\'on del problema}
%aca ponemos una interpretacion de lo que nos pide el enunciado y algunas aclaraciones de como vamos a encarar el problema.

\subsection{Resoluci\'on}
\subsubsection{Algoritmo}
%aca ponemos una descripcion de nuestro algorimtmo, presentamos la variables las estructuras y decimos que hacemos.

\subsubsection{Pseudoc\'odigo}
%aca va el pseudocodigo del problema.
\begin{algorithm}
\begin{algorithmic}
	\STATE input: $edificios$
%	\STATE $eventos$ = parseEdificios($edificios$)
	\WHILE{quedan edificios}
		\IF{empieza edificio}
			\STATE registro el edificio como abierto
			\IF{altura del edificio es mayor a la del contorno}
				\STATE agrego la altura del edificio al contorno
			\ENDIF 
		\ELSE
			\STATE saco al edificio de los abiertos
			\IF{este edificio le daba la altura al contorno}
				\STATE agrego la altura del edificio abierto que le siga en altura al contorno
			\ENDIF
		\ENDIF
	\ENDWHILE
	
	\RETURN contorno
\end{algorithmic}
\end{algorithm}

\subsection{Demostraci\'on}
%aca va la demostracion formal del problema refiriendonos al pseudocodigo o redefiniendo variables (definir todas las cosas de las que vamos a hablar).

\subsection{An\'alisis de complejidad}
%aca decimos cuanto cuesta cada parte del algoritmo y damos un valor final de la complejidad del algoritmo, ej O(logn).

\subsection{Test de complejidad}
%aca van los graficos y todos los testeos que hagamos para probar que en la practica el algoritmo cumple la complejidad que propusimos en el punto anterior

\subsection{Testing}
%aca ponermos todos nuestros casos bordes, como actua nuestro algoritmo en los casos particulares.































