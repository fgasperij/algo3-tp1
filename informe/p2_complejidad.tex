Para calcular la complejidad del algoritmo que implementamos primero debemos
explicar cómo resolvimos en la implementación el cáculo de maxAbierto($x$) con
una estructuras de datos conveniente que nos permite realizar todas las
operaciones necesarias sin superar la cota de complejidad requerida.
Contamos con un Multiconjunto, que llamaremos $abiertos$, que nos provee la
siguiente interfaz:
\begin{itemize}
	\item inserción: $O(log n)$
	\item borrado: $O(log n)$
	\item obtención del máximo: $O(log n)$
\end{itemize}
utilizaremos esta estructura de forma tal que calcular la función maxAbierto($x$)
sea equivalente a obtener el máximo de este multiconjunto. Para ello debemos
mantenerlo actualizado, esto implica que en cada iteración si el evento es de tipo
apertura lo agregamos a $abiertos$ (costo O($log n$)). 

La complejidad de este algoritmo puede calcularse como:
\begin{displaymath}
	O(extraer_eventos) + \#eventos * O(proximo_evento) + 
\end{displaymath}
