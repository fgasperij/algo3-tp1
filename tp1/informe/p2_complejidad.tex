Para calcular la complejidad del algoritmo que implementamos primero debemos
explicar cómo resolvimos en la implementación el cáculo de maxAbierto($x$) 
utilizando una estructuras de datos conveniente que nos permite realizar todas las
operaciones necesarias sin superar la cota de complejidad requerida.
Contamos con un Multiconjunto, que llamaremos $abiertos$, que nos provee la
siguiente interfaz:
\begin{itemize}
	\item inserción: $O(log n)$
	\item borrado: $O(log n)$
	\item obtención del máximo: $O(log n)$
\end{itemize}
utilizaremos esta estructura de forma tal que calcular la función maxAbierto($x$)
sea equivalente a obtener el máximo de este multiconjunto. Para ello debemos
mantenerlo actualizado, esto implica que en cada iteración si el evento es de tipo
apertura agregamos su altura a $abiertos$ (costo O($log n$)) y si es de cierre 
eliminamos sólo una vez la altura del mismo (costo O($log n$)). De esta forma
conocer la altura máxima de entre los edificios abiertos se reduce a encontrar el 
máximo dentro de $abiertos$ (costo O($log n$)).
Basándonos en el pseudocódigo presentado la complejidad de este algoritmo puede 
calcularse como:
\begin{equation}
\begin{split}
	O(extraer\_eventos) + \#eventos * O(proximo\_evento) + \\
	\#eventos\_de\_apertura * O(insertar\_en\_abiertos) + \\
	\#eventos\_de\_cierre * O(eliminar\_de\_abiertos) + \\
	\#contorno * (O(insertar\_en\_lista) + O(obtener\_maximo))
\end{split}
\end{equation}
\begin{itemize}
	\item extraer\_eventos: generamos 2 eventos por cada edificio y el costo
	de generar un evento es O(1). Por lo tanto, el costo total es O($2n$).
	\item proximo\_eventos: nosotros realizamos el orden primero y luego simplemente iteramos
	la estructura donde los guardamos, un Arreglo, cuyo costo de acceso eso $O(1)$. Por lo
	tanto, el costo se encuentra fuera del ciclo y es el costo de ordenar los eventos. Como
	las operaciones necesarias para decidir si un evento es mayor que otro están acotadas
	por una constante el costo de ordenar es igual que si fueran tipos primitivos. La función
	sort() de la STL nos garantiza una complejidad O($n log n$).
	\item \#eventos es exactamente igual a $2*n$ siendo $n$ la cantidad de edificios
	de la instancia. Tenemos una cantidad de eventos de apertura exactamente igual a la 
	cantidad de eventos de cierre: $\#eventos\_de\_apertura = \#eventos\_de\_cierre = n/2$
	\item insertar, eliminar y obtener el máximo de $abiertos$ ya vimos que tiene un costo
	de O($log n$). Como la cantidad de abiertos podemos fácilmente acotarla por $n$ ya que
	no pueden haber más edificios abiertos que el total de edificios que tenemos, éstas
	operaciones tienen costo O($log n$) siendo $n$ la cantidad de edificios.
	\item la cantidad de puntos del contorno podemos acotarla por la cantidad de eventos,
	cada evento podría llegar a representar un punto del contorno $\#contorno < 2n$.
\end{itemize}
Reemplazando en la ecuación anterior:
\begin{equation}
\begin{split}
	Costo = O(2n) + O(n\text{ }log\text{ } n) + 2n * O(1) + n * O(log\text{ } n) + n * O(log\text{ } n) + 2n * (O(1) + O(log\text{ } n))
\end{split}
\end{equation}
\begin{equation}
\begin{split}
	Costo = O(n) + O(n\text{ }log\text{ } n) + O(n) + O(n\text{ }log\text{ } n) + O(n\text{ }log\text{ } n) + (O(n) + O(n\text{ }log\text{ } n))
\end{split}
\end{equation}
\begin{equation}
\begin{split}
	Costo = O(n\text{ }log\text{ }n)
\end{split}
\end{equation}
